%%%%%%%% Miscellany packages %%%%%%%%%%%
\usepackage{ifthen}
\usepackage{url}
\usepackage{import}
\usepackage{comment}
\usepackage{soul} % to cross words
\usepackage{pifont} % for ding
\usepackage{epigraph} % for quotes
% to quote parts of citations
% \usepackage[autostyle]{csquotes}  
% for start of chapter 
\usepackage{lettrine}
%B format example
% \lettrine[lines=4]{\color{BrickRed}S}{tart} of the chapter
% for customized lists
\usepackage{enumerate}
\usepackage[shortlabels]{enumitem}
% use [start=0,label={(\bfseries A\arabic*):}]
% horizontal line across the page
\newcommand{\horz}{
\vspace{-.4in}
\begin{center}
\begin{tabular}{p{\textwidth}}\\
\hline
\end{tabular}
\end{center}
}
% Highlight a newly defined term
\newcommand{\newterm}[1]{{\bf #1}}
%%%%%%%%%%%%%%%%%%%%%%%%%%%%%%%%%%%%%%%%%%%%%%%%%%%%%%%%
%%%%%%%%%%%%% hyperref and its settings %%%%%%%%%%%%%%%%%%%%%%%%
\usepackage{hyperref}
%%%%% for metadata %%%%%%%%
\hypersetup{pdfauthor={Simone Maria Giancola},
	pdftitle={A roadmap to Message Passing methods for inference on Mixed Generalized Linear Models, with emphasis on Mixed Noiseless Phase Retrieval.},
	pdfsubject={MSc Thesis},
	pdfkeywords={Statistical Physics, Machine Learning, Generalized Linear Models, Approximate Message Passing}
	%pdfproducer={Latex with hyperref, or other system},
}
\hypersetup{colorlinks=false, urlcolor=magenta}
%%%%%%%%%%%%%%%%%%%%%%%%%%%%%%%%%%%%%%%%%%%%%%%%%%%%%%%
%%%%%%%%%%%%%%%%% geometry and its settings %%%%%%%%%%%%%%%%
\usepackage[a4paper, total={5in, 10in}]{geometry}
\savegeometry{origin}
\geometry{rmargin=2cm,lmargin=2cm}% for the title page
%%%%%%%%%%%%%%%%%%%%%%%%%%%%%%%%%%%%%%%%%%%%%%%%%%%%%%%%%%%%%
%%%%%%%%% bibliography management %%%%%%%%%%%%%%%%%%%%%%%%
\usepackage[
backend=biber,
style=authoryear-comp,
]{biblatex}
%%%%%%%%%%%%%%%%%%%%%%%%%%%%%%%%%%%%%%%%%%%%%%%%%%%%%%
%%%%%%%%% boxes management   %%%%%%%%%%%%%%%%%%%%%%%%
\usepackage{tcolorbox} % for colored boxes
\newcommand{\lawbox}[2]{\begin{tcolorbox}[colback=blue!5!white,colframe=blue!75!black,title=#1]
#2 \end{tcolorbox}}
\newcommand{\questionbox}[2]{\begin{tcolorbox}[colback=red!5!white,colframe=red!75!black,title=#1]
#2 \end{tcolorbox}}
\definecolor{amethyst}{rgb}{0.6, 0.4, 0.8}
\definecolor{brickred}{rgb}{0.8, 0.25, 0.33}
\newcommand{\ideabox}[2]{\begin{tcolorbox}[colback=amethyst!5!white,colframe=amethyst!75!black,title=#1]
#2 \end{tcolorbox}}
\newcommand{\summarybox}[1]{\begin{tcolorbox}[colback=black!5!white,colframe=black!75!black,title=Summary]
#1 \end{tcolorbox}}
\newcommand{\referencesbox}[1]{\begin{tcolorbox}[colback=green!5!white,colframe=green!75!black,title=Further References]
#1 \end{tcolorbox}}
\newcommand{\sectionstartbox}[1]{\begin{tcolorbox}[colback=gray!5!white,colframe=gray!75!black,title= In a nutshell]
#1 \end{tcolorbox}}
%%%%%%%%%%%%%%%%%%%%%%%%%%%%%%%%%%%%%%%%
%%%%%%%%% colors management %%%%%%%%%%%%%%%%%%%%%%%%
\usepackage{color}
\usepackage[dvipsnames]{xcolor}
% \definecolor{mydarkgreen}{RGB}{5,81,23}
%\definecolor{mydarkgreen}{RGB}{39,130,67}
\definecolor{mydarkred}{RGB}{192,25,25}
\definecolor{mydarkgreen}{RGB}{25,192,25}
\definecolor{mydarkblue}{RGB}{25,25,192}
\definecolor{blue-violet}{rgb}{0.54, 0.17, 0.89}
\newcommand{\red}{\color{mydarkred}}
\newcommand{\green}{\color{mydarkgreen}}
\newcommand{\blue}{\color{mydarkblue}}
\newcommand{\blueviolet}{\color{blue-violet}}
\newcommand{\cmark}{\green\ding{51}}%
\newcommand{\xmark}{\red\ding{55}}%
%%%%%%%%%%%%%%%%%%%%%%%%%%%%%%%%%%%%%%%%%%%%%%%%
%%%%%%%%% Productivity %%%%%%%%%%%%%%%%
% Here TODOs and whatnot
\usepackage[colorinlistoftodos,bordercolor=orange,backgroundcolor=orange!20,linecolor=orange,textsize=scriptsize]{todonotes}
% named command
\newcommand{\simone}[1]{\todo[inline]{{\textbf{Simone:} \emph{#1}}}}
% Generic TODO command. 
\newcommand{\TODO}[1]{\textcolor{red}{\textbf{TODO:}#1}}
\newcommand{\hotidea}{{\color{red}\bf HOT IDEA: }}
\newcommand{\done}{{\color{blue}\bf DONE }}
%%%%%%%%%%%%%%%%%%%%%%%%%%%%%%%%%%%%%%%
%%%%%%%% packages for math stuff %%%%%%%%%%%%%%%%
% to avoid too many math alphabets
\newcommand{\hmmax}{0} 
\newcommand{\bmmax}{0}
\usepackage{mathrsfs}
\usepackage{annotate-equations}
\renewcommand{\eqnhighlightheight}{\vphantom{\hat{H}}\mathstrut} % setting for annotate
\usepackage{amsfonts}
\usepackage{mathtools}
\usepackage{amssymb}
\usepackage{bm} % for bold math
\usepackage{amsthm}
\usepackage{upgreek}
\usepackage{bbm} % for indicator function
\usepackage{cleveref}
\usepackage{nicefrac}
\usepackage{fixdif}   % for differential operator
%\usepackage{physics}
\usepackage{showlabels} % to show label equations
\usepackage{thmtools} % for list of theorems
\usepackage{thm-restate} % to restate theorems
\usepackage{etoolbox} % for list of theorems
\usepackage{mathabx} % other fonts
\usepackage{siunitx} % standard unit symbols
%%%%%%%%%%%%%%%%%%%%%%%%%%%%%%%%%%%%%%%
%%%%%%%%%%% packages for images and figures %%%%%%%%%%%%%%%%
\usepackage{svg}
\usepackage{subcaption}
\usepackage{adjustbox}
\usepackage{wrapfig}
% Mark sections of captions for referring to divisions of figures
\newcommand{\figleft}{{\em (Left)}}
\newcommand{\figcenter}{{\em (Center)}}
\newcommand{\figright}{{\em (Right)}}
\newcommand{\figtop}{{\em (Top)}}
\newcommand{\figbottom}{{\em (Bottom)}}
\newcommand{\captiona}{{\em (a)}}
\newcommand{\captionb}{{\em (b)}}
\newcommand{\captionc}{{\em (c)}}
\newcommand{\captiond}{{\em (d)}}
%%%%%%%%%%%%%%%%%%%%%%%%%%%%%%%%%%%%%%%%%%%%%%%%%%%%%%%%
%%%%%%%%% Tikz and its libraries  %%%%%%%%%%%%%%%%%%%%%%%%
\usepackage{tikz}
\usetikzlibrary{shapes}
\usetikzlibrary{shapes.arrows, decorations.pathmorphing}
\usetikzlibrary{patterns}
\usetikzlibrary{hobby} % for ..
\usetikzlibrary{arrows.meta} % to control arrow size
\tikzset{>={Latex[length=4,width=4]}} % for LaTeX arrow head
\usetikzlibrary{calc,intersections,decorations.markings}
%%%%%%%%%%%%%%%%%%%%%%%%%%%%%%%%%%%%%
%%%%%%%%% pgf plots and its libraries %%%%%%%%%%%%%%%%%%%%%%%%
\usepackage{pgfplots}
\usepgfplotslibrary{fillbetween}
%%%%%%%%%%%%%%%%%%%%%%%%%%%%%%%%%%%%%%%%%%%%%%%%%%
%%%%%%%%% algorithm and its settings %%%%%%%%%%%%%%%%%%%%%%%%
\usepackage{algorithm}
\usepackage{algpseudocode}
% \usepackage{algorithmic}
% algname
\newcommand{\algname}[1]{{\color{green}\small\textsf#1}\xspace}
\newcommand{\algnamesmall}[1]{{\color{green}\scriptsize\textsf#1}\xspace}
\newcommand{\algnametiny}[1]{{\color{green}\tiny\textsf#1}\xspace}
%%%%%%%%%%%%%%%%%%%%%%%%%%%%%%%%%%%%%%%%%%%%%%%

%%%%%%%%% added lists at the beginning %%%%%%%%%%%%
% Acronyms list and Symbols list
\usepackage[acronym, toc, section=section]{glossaries}
% \makeglossaries
% \newacronym{gcd}{GCD}{Greatest Common Divisor}
% \printglossary[type=\acronymtype]
% nomenclature
\usepackage{nomencl}
%%%%%%%%%%%%%%%%%%%%%%%%%%%%%%%%%%%%%%%%%%%%%%%%

%%%% REFERENCE SYSTEM %%%%%%%%%%%%%%%% Figure reference, lower-case.
\def\figref#1{figure~\ref{#1}}
% Figure reference, capital. For start of sentence
\def\Figref#1{Figure~\ref{#1}}
\def\twofigref#1#2{figures \ref{#1} and \ref{#2}}
\def\quadfigref#1#2#3#4{figures \ref{#1}, \ref{#2}, \ref{#3} and \ref{#4}}
% Section reference, lower-case.
\def\secref#1{section~\ref{#1}}
% Section reference, capital.
\def\Secref#1{Section~\ref{#1}}
% Reference to two sections.
\def\twosecrefs#1#2{sections \ref{#1} and \ref{#2}}
% Reference to three sections.
\def\secrefs#1#2#3{sections \ref{#1}, \ref{#2} and \ref{#3}}
% Reference to an equation, lower-case.
\def\eqref#1{equation~\ref{#1}}
% Reference to an equation, upper case
\def\Eqref#1{Equation~\ref{#1}}
% A raw reference to an equation---avoid using if possible
\def\plaineqref#1{\ref{#1}}
% Reference to a chapter, lower-case.
\def\chapref#1{chapter~\ref{#1}}
% Reference to an equation, upper case.
\def\Chapref#1{Chapter~\ref{#1}}
% Reference to a range of chapters
\def\rangechapref#1#2{chapters\ref{#1}--\ref{#2}}
% Reference to an algorithm, lower-case.
\def\algref#1{algorithm~\ref{#1}}
% Reference to an algorithm, upper case.
\def\Algref#1{Algorithm~\ref{#1}}
\def\twoalgref#1#2{algorithms \ref{#1} and \ref{#2}}
\def\Twoalgref#1#2{Algorithms \ref{#1} and \ref{#2}}
% Reference to a part, lower case
\def\partref#1{part~\ref{#1}}
% Reference to a part, upper case
\def\Partref#1{Part~\ref{#1}}
\def\twopartref#1#2{parts \ref{#1} and \ref{#2}}
%%%%%%%%%%%%%%%%%%%%%%%%%%%%%%%%%%%%%%%%%%%%%%%%%%%%%%%%%%%%%%%
%%%%%%% Renewed commands %%%%%%%%
% \renewcommand{\thechapter}{\Roman{chapter}} 
\renewcommand{\thesubsection}{\thesection.\Roman{subsection}}
\numberwithin{equation}{section}

%%%%%%%% Bibliography Management %%%%%%%%%%%%%%%%
% \usepackage[
% backend=biber,
% style=alphabetic,
% ]{biblatex}

% \addbibresource{bib.bib}
%%%%%%%%%%%%%%%%%%%%%%%%%%%%%%%%%%%%%%%%

%%%%%%%% DEPRECATED %%%%%%%%%%%%%%%%%%%%%%%%%%%%%%%%
% \newenvironment{abstractmod}%
%     {\cleardoublepage\thispagestyle{empty}\null\vfill\begin{center}%
%     \bfseries Abstract\end{center}}%
%     {\vfill\null}
%     \newenvironment{acknowledgements}%
%     {\cleardoublepage\thispagestyle{empty}\null\vfill\begin{center}%
%     \bfseries Acknowledgements\end{center}}%
%     {\vfill\null}
% \usepackage{verbatim} % for commenting multiple lines
% \declaretheorem{theorem}
%%%%% for differentials
%\usepackage{fontsetup}
%%% for partial differentiatio
% checkmark and crossmark
%\newcommand{\myred}[1]{{\color{RedOrange}#1}} % RGB ‎255, 83, 73    #ff5349
%\newcommand{\mygreen}[1]{{\color{ForestGreen}#1}} % RGB 34,139,34
%\newcommand{\myblue}[1]{{\color{NavyBlue}#1}}
% strange stuff
%\usepackage[round]{natbib}
%\setcitestyle{authoryear,round,citesep={;},aysep={,},yysep={;}}
%\renewcommand{\bibname}{References}
%\renewcommand{\bibsection}{\subsubsection*{\bibname}}
%%%%%%%%%%%%%%%%%%%%%%%%%%%%%%%%%%%%%%%%%%%%%%%%%%%%%%%%%%%%%%%%



